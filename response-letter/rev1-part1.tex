\subsection{Part 1: Typographical / Grammatical / Stylistic revisions}
\newcounter{rev1-part1}
\setcounter{rev1-part1}{1}

\textbf{Comment \arabic{rev1-part1}} \\
On page 2, sentence 2 of section 1.2 (ISTK Model) has issues with verb tenses that do not match (We took vs.\ initialize).

\textbf{Response \arabic{rev1-part1}} \\
This was changed so that tenses agree.
\stepcounter{rev1-part1}

\textbf{Comment \arabic{rev1-part1}} \\
On page 3, near the end of paragraph 2, of section 1.3, there is a sentence that begins ``The second term of Equation 1 accounts for the conjugate of `believing the rumor.'\,'' I don't believe that ``conjugate'' is the appropriate word here. Perhaps ``complement'' or ``opposite'' would be more accurate?

\textbf{Response \arabic{rev1-part1}} \\
The word ``conjugate'' is now ``complement.''
\stepcounter{rev1-part1}

\textbf{Comment \arabic{rev1-part1}} \\
The axes on all graphs in this paper (as well as the traces of the data being plotted) were too thin.
They barely showed up on the hard copy of the manuscript that I printed for myself.
They were also rather faint on the screen rendering of the electronic copy.
The vertical and horizontal grid lines on the figures are the hardest of all of these features to see.
If the authors could increase the weight of the axes, the traces of the graphs plotted on them, and the grid lines the figures might be easier to read.

\textbf{Response \arabic{rev1-part1}} \\
We darkened and thickened the lines and axes.
\stepcounter{rev1-part1}

\textbf{Comment \arabic{rev1-part1}} \\
The axis label for the independent axis on figure 2 says $ \alpha 1 $ value (no subscript), when the parameter in question is referred to as $ \alpha_1 $ (with a subscript) in the text.
These should be consistent.

\textbf{Response \arabic{rev1-part1}} \\
This is fixed to be correctly subscripted (note the figure is now Figure 3).
\stepcounter{rev1-part1}
