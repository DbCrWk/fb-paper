\subsection{Part 2: General Recommendations}
\textbf{Comment} \\
It is my opinion that the manuscript ``Rumors with Personality,'' by Dabke and Arroyo (supervised by Layton) is of sufficient quality and interest for publication in SIURO.
While I've found no flaws in the manuscript that warrant major revision, I'd like to make some suggestions for points to make in the conclusion.
The authors point out that the richest and least trivial dynamical behavior seems to be present in agent based models with feature vectors that describe both agents and rumors.
These models, in many instances, exhibit rumors that die out before all members of the network (even if it is fully connected) observe it.
The authors already speculate that it might be possible to engineer a perfect rumor that will fully saturate the network.
It's not clear to me if this engineering would be more dependent upon the feature vectors of the agents in the network or upon the topology and architecture of the network itself.
Some clarification (or at least mention) of this might be helpful.
In any case, other questions for future study that might be of equal interest would be is it possible to design a network topology that is somehow not so restrictive that it still allows agents to interact in ways the people expect of social networks yet it also effectively dampens the spread of as many rumors as possible.
Likewise, would it be possible for an agent within a network to engineer their own feature vector (even if it means lying about their personal attributes when creating their account) in a way that inoculates them against the spread of a majority (or at least a plurality) of the prevailing rumors.

\textbf{Response} \\
Please see new final paragraph in discussion, we have more fully discussed that there.
Essentially we conclude that that we show the topology is not restrictive, in fact it is the characteristics of the agents and their arrangement.
In addition, we mention that it isn't a factor of the characteristics of one individual, but more to do with the similarity of the individuals in a population.
