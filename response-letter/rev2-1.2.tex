\subsection{Section 1.2}
\newcounter{rev2-1.2}
\setcounter{rev2-1.2}{1}

\textbf{Comment \arabic{rev2-1.2}} \\
Often language is used about the personality of the rumor, when I think it the personality of the individual instead.

\textbf{Response \arabic{rev2-1.2}} \\
In the feature-vector, agent-based model, we equip rumors with a feature vector.
This feature vector parallels the demographics (encoded as vectors) for each agent in the model.
Therefore, we define this feature vector as the ``personality'' of the rumor.
The rationale behind this name was two-fold: we believe that it provides a succint link from the feature vector of the rumor to the corresponding characteristic vector of the agents; we also believe that this term incites intrigue and colorfully illustrates the importance and dynamism of the rumor.
The ``personality'' of the rumor is our term for its characteristic feature vector.
We kept the terms and clarified its first usage in this paper, which is in this section, in the second paragraph.
\stepcounter{rev2-1.2}


\textbf{Comment \arabic{rev2-1.2}} \\
For example in the first paragraph: ``Instead of assuming that every individual is equally likely to spread any rumor, we assumed that the rumor's personality and the demographic information of each individual affected the likelihood of the rumor to spread.''
I would argue that the rumor has some characteristics perhaps, but not a personality.

\textbf{Response \arabic{rev2-1.2}} \\
(See above comment/response)
\stepcounter{rev2-1.2}

\textbf{Comment \arabic{rev2-1.2}} \\
A few sentences later you say: ``The similarity of the rumor to the individual\textellipsis'' again I would argue that rumors and individuals are not similar.
Also, it appears that in your model here that you spend all of your effort on modeling the individuals (either as ``ignorant'', ``spreader'', ``stifler'' or ``knowledgeable'') and didn't address the characteristics of the rumor.

\textbf{Response \arabic{rev2-1.2}} \\
We clarified that this is the ``The similarity of the rumor's personality to that of the individual's'' to address this relaxed use of the word ``similarity.''
Overall, when we discuss the rumor, since it has a ``personality'', i.e.\ a feature vector that is analagous to those of the agents, we felt comfortable anthropomorphizing the rumor.
We have minimized this effect.
Furthermore, we made our language more concise and direct when discussing the rumor, so as to highlight the importance and significance of the rumor's personality.
\stepcounter{rev2-1.2}
