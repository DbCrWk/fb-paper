% BG-title.tex 
% 


\title{Rumors with Personality: \\ \large A Differential and Agent-Based Model of Information Spread through Networks } 

%\author[1]{Devavrat V. Dabke \thanks{Email: d.d@duke.edu}}
%\author[1]{Eva E. Arroyo \thanks{Email: eva.arroyo@duke.edu}} 
%\affil[1]{ Duke University, Durham, NC } 

\author{Devavrat V. Dabke\thanks{Email: d.d@duke.edu, co-author}\qquad Eva E. Arroyo\thanks{Email: eva.arroyo@duke.edu, co-author}\\Duke University, Durham, NC}

\date{\parbox{\linewidth}{\centering%
  Supervised by: \\[1.1em]
  Prof. Anita T. Layton \\
Depts. of Mathematics and Biomedical Engineering \\
Duke University, Durham, NC \\
alayton@math.duke.edu
  }
  }

\maketitle 

%%%% LEFT THE OLD ABSTRACT HERE %%%%
%\begin{abstract} { In this paper, a variation of the SIR infectious disease spread model--the ``ISTK" model--was used to model the spread of a rumor in a given social network. Initially, we modeled with ordinary differential equations to assess the spread of a rumor in face-to-face interaction between groups of people. Then, we modified this approach to model the spread of a rumor over Facebook. We then used an agent-based model with a network given by a Stanford dataset of model Facebook connections to see how the addition of a social network affected the spread of viral information. Finally, by simulating demographics of each agent-based on the Stanford dataset, we initialize feature vectors for each member of the population and for the rumor to see how the characteristic of a rumor affects its spread through a network. Our results showed that though the rumor always spreads through the population in the differential and agent-based models without features, once the features are included the rumor necessarily dies out.}

\begin{abstract}

We constructed the ``ISTK'' model to approximate the spread of viral information, a \textit{rumor}, through a given (social) network. Initially, we used a set of ordinary differential equations to assess the spread of a rumor in face-to-face interactions in a homogenous population. Our next model translated this system into an equivalent stochastic agent-based model. We then incorporated a network based off of a representative Facebook dataset. Our last model considered \textit{features}: demographic information that characterized individuals in our representative population. We also generated a feature vector for the rumor in order to simulate its ``personality.'' An increase in the average similarity of the rumor to the population resulted in increased propagation through the network. Our results showed that incorporating the structure of a network alters the behavior of the rumor as it spreads across the population, while preserving the steady-states. However, the addition of feature vectors prevents the rumor from saturating the network. Our agent-based, feature-equipped ISTK model provides a more realistic mechanism to account for social behaviors, thus allowing for a more precise model of the dynamics of rumor spread through networks.

\end{abstract}