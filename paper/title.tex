\title{Rumors with Personality: \\ \large Differential and Agent-Based Models of Information Spread through Networks }

\author{Devavrat V. Dabke\thanks{Email: d.d@duke.edu}\qquad Eva E. Arroyo\thanks{Email: eva.arroyo@duke.edu}\\Duke University, Durham, NC}

\date{\parbox{\linewidth}{\centering%
  Supervised by: \\[1.1em]
  Prof.\ Anita T. Layton \\
Depts.\ of Mathematics and Biomedical Engineering \\
Duke University, Durham, NC \\
alayton@math.duke.edu
  }
  }

\maketitle

\begin{abstract}

We devised the ``ISTK'' model to approximate the spread of viral information---a \textit{rumor}---through a given (social) network.
For our first version of the model, we used a set of ordinary differential equations to assess the spread of a rumor in face-to-face interactions in a homogenous population.
Our second model translated this system into an equivalent stochastic agent-based model, but also incorporated a network that encodes the relationship between individuals.
Our third model considered \textit{features}: demographic information that characterizes individuals in our representative population.
We also generated a feature vector for the rumor, its \textit{personality}, to simulate the targeting of viral information.
Our results showed that incorporating the structure of a network alters the dynamics of a rumor's spread, but preserves steady states.
However, the addition of a feature vector greatly influenced the rumor's spread through the network; the rumor's ability to spread across the network was positively correlated with its ``similarity'' to the individuals.
This final agent-based, feature-vector ISTK model provides a more realistic mechanism to account for social behaviors, thus permitting more precise study of the dynamics of rumor spread through networks.

\end{abstract}
