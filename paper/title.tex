\title{Rumors with Personality: \\ \large A Differential and Agent-Based Model of Information Spread through Networks }

\author{Devavrat V. Dabke\thanks{Email: d.d@duke.edu, co-author}\qquad Eva E. Arroyo\thanks{Email: eva.arroyo@duke.edu, co-author}\\Duke University, Durham, NC}

\date{\parbox{\linewidth}{\centering%
  Supervised by: \\[1.1em]
  Prof.\ Anita T. Layton \\
Depts.\ of Mathematics and Biomedical Engineering \\
Duke University, Durham, NC \\
alayton@math.duke.edu
  }
  }

\maketitle

\begin{abstract}

We constructed the ``ISTK'' model to approximate the spread of viral information---a \textit{rumor}---through a given (social) network.
Initially, we used a set of ordinary differential equations to assess the spread of a rumor in face-to-face interactions in a homogenous population.
Our next model translated this system into an equivalent stochastic agent-based model.
We then incorporated a network based off of a representative Facebook dataset.
Our second model considered \textit{features}: demographic information that characterized individuals in our representative population.
We also generated a feature vector for the rumor in order to simulate its ``personality.''
An increase in the average similarity of the rumor to the population resulted in increased propagation through the network.
Our results showed that incorporating the structure of a network alters the behavior of the rumor as it spreads across the population, while preserving steady states.
However, the addition of feature vectors prevents the rumor from saturating the network.
Our agent-based, feature-equipped ISTK model provides a more realistic mechanism to account for social behaviors, thus allowing for a more precise model of the dynamics of rumor spread through networks.

\end{abstract}
