\section{Discussion }
\label{sec:discussion}

There was relatively little difference between the end states of the differential and simple agent-based models, despite the fact that the former aggregates the population and the latter provides more granularity.
As previously noted by Chierchetti et al.
\cite{chierchetti-2010}, in a fully-connected network with push-pull interactions, a rumor will spread to the majority of the population with high probability.
We came to similar conclusions as this previous study's findings, though our model had a fully-connected network and assumed different interactions: individuals only had the opportunity to interact with the same individuals at every time step, as opposed to choosing new ``partners'' each time~\cite{chierchetti-2010}.
In the agent-based model, not every individual learned about the rumor, and the addition of some structured social network delayed rumor spread.
That is to say, rumors must diffuse through a complex network in order to have large-scale effects on the population.
Thus, the curves are less dramatic, change more gradually, and there is no guarantee every individual will hear the rumor: by the end of the $ 22 $ days essentially none of the population remains ignorant in the differential model, whereas in the agent-based model $ 2.8\% $ of the population remains ignorant.
However, the trajectories of the two models are qualitatively similar, suggesting that the agent-based model tends to a vanishing of an ignorant population, save for a small connected subnetwork.
Just like the claim so well supported in push models, eventually there is a high probability all individuals will hear the rumor~\cite{pittel-1987, angelopoulos-2009}.

The incorporation of feature vectors in the agent-based model changes the overall spread of the rumor, since spread of the rumor is a factor of both the similarity of individuals to each other, and similarity of the rumor to each individual.
As indicated by our results, even where the rumor spreads, individuals become stiflers so quickly that the rumor dies out before reaching a large proportion of the population.
Perhaps, if the individuals in our network cluster based on features, it would explain how a rumor could die out trying to navigate several disjoint subnetworks.
This behavior is familiar to anyone who has been on social media, and had friends who relentlessly post stories that bear no significance to their personal beliefs or preferences.
We speculate that in a community with many highly similar individuals one could much more easily engineer a rumor to spread through the whole network.
However, an individual hearing the rumor has less to do with their individual traits, than the similarity of individuals to each other in the population.
Since the simple, agent-based model still spread the rumor to (essentially) every individual, the topology of the network is probably less important in preventing rumor spread.
Inoculation against hearing the rumor seems more a factor of the general dissimilarity of individuals to each other in the population.
We suspect that the inevitable ``death'' of our rumors may be due to a population of individuals with a great variety of different feature vectors.

Most notably, this is a great departure from the original results presented in the Daley-Kendall (DK) Model.

\section{Conclusions}
\label{sec:conclusions}
Since the rumor tends to spread rapidly at the start of the simulation (resulting in a corresponding boost in the stifler population), these results inspire the consideration of different network configurations.
A rumor spreads rather more quickly in preferentially connected real-world graphs than in common theoretical mathematical graphs~\cite{doerr-2012}, however, In our case, even the ``best-performing'' rumor---one that maximized spread---still died out.
Tt would seem that our diverse interests and demographics prevent global conquest by any one rumor.
That being said, it might be possible to construct a network topology that depended on the features, which might promote better spread, or even that would saturate the network.
Alternatively, it might be possible to place strategic ``blockers'': individuals who inhibited the spread of rumors between subnetworks.
In all, it would seem as though our model---in part thanks to the advent of increased computing power for simulations---can begin to unravel the nuances and intricacies of information spread through a social network.
By arriving at a model that uses feature vectors and graphs, we have greater control and specificity in looking at the spread of viral information, possibly leading us to mathematically ``perfect'' viral information.
