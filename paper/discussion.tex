\section{Discussion }
\label{sec:discussion}

There was relatively little difference between the end states of the differential and agent-based models, despite the fact that the former aggregates the population and the latter provides more granularity.
As previously noted by Chierchetti et al.
\cite{chierchetti-2010}, in a fully-connected network with push-pull interactions, a rumor will spread to the majority of the population with high probability.
Though our model has a significantly different setup, we came to similar conclusions as this previous study's findings, though our model had a fully-connected network and assumed different interactions: individuals only had the opportunity to interact with the same individuals at every time-step, as opposed to choosing new ``partners'' each time~\cite{chierchetti-2010}.
In the agent-based model, not every individual learned about the rumor, and the addition of some structured social network causes a delay in rumor spread.
That is to say, the effects on one cluster are not immediately transferred to another cluster, as effects upon individuals must travel through the other individuals in a complex network in order to have large-scale effects on the population.
Thus, the curves are less dramatic, change more gradually, and there is no guarantee every individual will hear the rumor: by the end of the $ 22 $ days essentially none of the population remains ignorant in the differential model, whereas in the agent-based model $ 2.8\% $ of the population remains ignorant.
However, the trajectories of the two models are qualitatively similar, suggesting that the agent-based model tends to a vanishing of an ignorant population, save for a small connected subnetwork.
Just like the claim so well supported in push models, eventually there is a high probability all individuals will hear the rumor~\cite{pittel-1987, angelopoulos-2009}.

The incorporation of Feature vectors in the agent-based model changes the overall spread of the rumor.
Even in the case in which most people hear the rumor (the most similar feature vector) there remains a significant population that never hears the rumor, a factor of the similarity between individuals and the rumor.
Facebook friendships are a relationship that we take here to model real-world social networks.
However, Facebook friends are likely to be more superficial.
In fact, the average number of Facebook friends is 338~\cite{smith-2014}, yet Dunbar's number suggests that humans cannot maintain more than 150 relationships due to neocortex size~\cite{dunbar-1992}.
The social network we use shows the spread of a rumor in people who are not necessarily close, but do interact.
Perhaps the feature-vector-based spread in a Facebook network is less effective in spreading the rumor due to this superficiality of relationships.
Perhaps if individuals in a Facebook network cluster based on features, it would explain how a rumor could die out trying to navigate a dissimilar subnetwork.
As indicated by our results, even where the rumor spreads, individuals become stiflers so quickly that the rumor dies out before reaching a large proportion of the population.
This behavior is familiar to anyone who has been on social media, and had friends who relentlessly post stories that bear no significance to their personal beliefs or preferences.


Perhaps networks with clusters of similar people (by their feature vectors) would aid in the rapid transmission of a rumor across a network.
In the Feature vector model, spread of the rumor is a factor of both the similarity of individuals to each other, and similarity of the rumor to each individual.
We speculate that in a community with many highly similar individuals one could much more easily engineer a rumor to spread through the whole network.
However, an individual hearing the rumor less to do with their individual traits, than the similarity of individuals to each other in the population.
We show that there is nothing in the topology of the network that prevents rumor spread in the simple agent based model, so inoculation against hearing the rumor is a factor of the general disimilarity of individuals to each other in the population.
We suspect that the inevitable ``death'' of our rumors may be due to a population of individuals with heterogeneous feature vectors.
Future models should investigate how the similarity of individuals' feature vectors impacts the spread of any rumor.


\section{Conclusions}
\label{sec:conclusions}
Since the rumor tends to spread rapidly at the start of the simulation (resulting in a corresponding boost in the stifler population), these results inspire the consideration of different network configurations.
A rumor spreads rather more quickly in preferentially connected real-world graphs than in common theoretical mathematical graphs~\cite{doerr-2012}, however, In our case, even the ``best-performing'' rumor---one that maximized spread---still died out.
However, it may be possible to engineer a rumor that saturates the network.
In all, it would seem as though our model---in part thanks to the advent of increased computing power for simulations---can begin to unravel the nuances and intricacies of information spread through a social network.
By arriving at a model that uses feature vectors and graphs, we have greater control and specificity in looking at the spread of viral information, possibly leading us to mathematically ``perfect'' viral information.
