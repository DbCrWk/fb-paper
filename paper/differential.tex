\section{Differential Model }
\label{sec:diffmodel}

\subsection{Modeling Applications}
\label{subsec:diffmodeapp}

<<<<<<< HEAD
In this first model, we examined how parameters based on face-to-face interaction had an impact on rumor spread versus interaction over a network (described by an adjacency matrix).
=======
Initially, we solved the differential model in order to compare a continuous model to a stochastic agent-based model.
We examined how parameters based on face-to-face interaction had an impact on rumor spread versus interaction over a network (described by an adjacency matrix).
>>>>>>> origin/master
Specifically, we used parameters for a model on consumer goods, in order to examine how long it took the rumor to reach a significant proportion (90\%) of the population and the effect on the amount of time until steady states were reached with perturbations in initial parameters.

\subsection{Estimating Parameters}
\label{subsec:diffmodeeparam}

The parameters necessary to estimate were \textit{credibility}, the
\textit{loss of novelty}, and the \textit{number of \textbf{close} interactions} for an individual.
Using consumer statistics on perceptions of reliability of information from different sources, we initially estimated the credibility $ c = 2.8/7 $~\cite{kamins-1997}.
Estimations in number of close contacts varied from $ 12 $--$ 26 $ people per day, varying based on age~\cite{cahill-1996, mossong-2008, edmunds-2006}.
We took the average number of close contacts to be $ 22 $ (i.e. $ \tau = 22 $, which determines our parameter of interaction $ l $).
Although the differential model itself does not change with the medium of Facebook, the meaning of $ \tau $ changes.
Instead of $ 22 $ close interactions per day, we selected an appropriate analog, in that we assumed the average individual reads approximately $ 22 $ posts per day.
We estimated the value representing loss of novelty at $ \alpha_1 = .01 $, and  $ \alpha_2 = .02 $, since the spreaders will have a stronger effect on the stiflers.
\
In all cases, these were the ``baseline'' parameters, and were only modified for the Feature vector model, in which case parameters $ c $ and $ \alpha_1 $  were based on the features vectors of agents, and rumors.
Sensitivity analyses for $ c $, and $ \delta $ were run on the interquartile ranges of the studies on which they were based.
$ \alpha_1 $ was an approximated variable, so we simply run as small of an $ \alpha_1 $ variable that the differential model was capable of processing, up to an $ \alpha_1 $ value of .25.
