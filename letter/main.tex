\documentclass{letter}
\usepackage{hyperref}
% Palatino for rm and math | Helvetica for ss | Courier for tt
\usepackage{mathpazo} % math & rm
\linespread{1.05}        % Palatino needs more leading (space between lines)
\usepackage[scaled]{helvet} % ss
\usepackage{courier} % tt
\normalfont
\usepackage[T1]{fontenc}

\usepackage[a4paper,left=2.5cm, right=2.5cm, top=2.5cm, bottom=2.5cm]{geometry}

\makeatletter
\renewcommand{\closing}[1]{\par\nobreak\vspace{\parskip}%
  \stopbreaks
  \noindent
  \ifx\@empty\fromaddress\else
  \hspace*{\longindentation}\fi
  \parbox{\indentedwidth}{\raggedright
       \ignorespaces #1\\[2\medskipamount]%
       \ifx\@empty\fromsig
           \fromname
       \else \fromsig \fi\strut}%
   \par}
\makeatother

\signature{Eva E. Arroyo \qquad Devavrat V. Dabke}
\address{Duke University \\
Campus Box 97110 \\
Durham, NC 27708}

\longindentation=0pt

\begin{document}

\begin{letter}{Dr. Luis Melara\\
Editor-in-Chief\\
SIAM Undergraduate Research Online\\}
\opening{To Dr. Melara:}

We are herewith submitting our paper entitled ``Rumors with Personality: A Differential and Agent-Based Model of Information Spread through Networks'' for consideration for publication in \textit{SIAM Undergraduate Research Online.} We are two undergraduate students attending Duke University, and were mentored on this project by Prof. Anita Layton, who is a professor in the Departments of Mathematics and Biomedical Engineering.

In this manuscript, we model rumor spread with both differential and agent-based models. The first part of our investigation assumes homogenous populations; when we introduce a network (i.e. only allowing spread of information through ``connected'' people), we found an appreciable change in the dynamics of the spread, but both ultimately yielded similar end states. We also include a sensitivity analysis for our differential model. The second part of our investigation incorporates the impact of features: self-recognized groups an individual belongs to, such as education level, age, and gender. These factors fundamentally alter the dynamics of rumor spread, as well as their terminal states.

We believe that this submission is appropriate for \textit{SIAM Undergraduate Research Online} as it is a mathematical model completed by undergraduates that considers the realistic social dynamics that dictate rumor spread. This paper has not been presented in any form at a conference.

We greatly appreciate your consideration of our submission
\vspace{24pt}
\closing{Sincerely,}

\end{letter}
\end{document}